%% ----------------------------------------------------------------
%% PSN.tex
%% ---------------------------------------------------------------- 


\chapter{Purposive Social Network} \label{Chapter:Purposive Social Network}

A community is created when people collectively create and share information providing a common source of knowledge and WWW provides an easy platform for people to come together and create a knowledgebase by crowdsourcing. Wikipedia is a good example where thousands of people come together and create and edit a shared knowledgebase. The most efficient online communities depend on user participations, easy moderation, interest in topic and a common tie between communitiesÕ members.

In this report a question and answer forum for the programmer, StackOverflow, is studied where people with a common interest come together and solves problems, and create a self-sustaining and self-regulating purposive social network.

\section{What is Purposive Social Network}

There are many reasons to join and form communities; human are social creature and forming a social tie is a natural instinct. In the online world, where geographical boundaries are dissolved and people from any part of the world can connect together, and also have a certain degree of anonymity and privacy, people can reach to others with similar background and interest.

A network with a purpose, as name suggests, is created when people come together with a common objective to get information, build knowledgebase, solve problems or achieve some common goal.

A purposive social network can be typically categorized into following types. It should be noted that a community can be of one or many categorizes mentioned below \cite{backstrom2006group}.


\subsection{Information Based Community}


They are communities that share some particular type of information or resources, communities where people come to share their particular knowledge and gather more information or news on any matter by IRC channel, forums or just mailing list is one of the examples. This kind of community generally has many new people joining to get some information but the rate of returning back is low \cite{tantipathananandh2007}.


\subsection{Interest Based Community}


They are communities where users come together to share their hobbies or interests and form a strong connection with each other. The number of people in the network is generally stable and they actively participate in any discussion. Online role-playing and gaming community is one example of this type of community \cite{newman2004detecting}.


\subsection{Expert Based Community}


They are formed mostly by experts in their field in order to discuss a common matter or to solve a common problem. Mostly academics or experts in a field form this kind of community.


\subsection{Location Based Community}

They are communities where people come together for a geographical reason and where proximity is important. These types of communities form because of social issues concerning a region: neighbourhood watch, Yoga and jogging groups or support groups are examples of this type of community \cite{smith2008social}.


\section{Characteristics of Purposive Social Network}

The main characteristics of a purposive community are same as any other community is the social network structure but some attributes make them different . These special characteristics of purposive social network are as follows \cite{katz2004network}:


\subsection{Community Size}

This type of community is created by people with similar objective and purpose and it does not have a large number of people but a small amount of people coming together to share knowledge and solve a problem.


\subsection{Focused Interest}

Small groups of people with shared interest come together to solve a very specific problem and create a focused knowledge.

\subsection{Direct Communication}

Each individual member in the community share information and communicate directly with other members, they communicate directly with one another.


\subsection{Active Participation}

Since purposive communities have focused goals and interest and small number is people, each member actively participates to solve the common problem.

\subsection{Short Lifespan}

The purposive micro community has a focused interest and purpose and it exist for a small period of time and once the problem is solved, the community dissipates.


\subsection{Strong Incentive}

Since the micro community exist for a small amount of time and only interested people participate focusing on solving a particular problem, strong incentive exists to motivate people to join and participate.


\section{Benefits of Purposive Social Network}

People are spending more and more time online, they connect with their friends and families, do their work and shopping online and use other peopleÕs recommendation and experience to solve their own problem. The social network analysis proves the six-degree of separation theory, that each of us is connected to any random person in the world through the right six people we know. But in the world of blogs, forums and messaging board this barrier is broken. One does not have to know anyone personally, professionally or at all to interact, comment or reply to one another \cite{monge2003theories}. The main motive for people to join such communities is discussed below.


\subsection{Information Exchange and Self-interest}

With people spending more and more of their time on the web, this became a place where they can discuss ideas, problems, issues and, being a social creature, they would like to do it with like-minded people with similar interest and issues \cite{flake2002self}.

Online communities are the best place to share and exchange information because it is available to everyone, anywhere in the world, people come together and join a forum or a discussion board to share knowledge about their hobbies, health or politics. People share different kind of information, from restaurants at Yell.com to websites in LiveJournal, two completely different websites for different purposes take advantage of user needs.

In the area of E-Government, the government (e.g. USA has data.gov \footnote{\url{http://www.data.gov/}}, UK has data.gov.uk \footnote{\url{http://www.data.gov.uk/}}) is publishing statistical and other Public Sector Information (or Open Government Data, OGD). People with different objectives come together and use this open government data and collaborate to prevent crime using crime statistic of their geographic area or create support groups based on the health and morbidity statistic of their nearby hospitals, websites like FixMyTransport \footnote{\url{www.fixmytransport.com/}}, PatientsLikeMe \footnote{\url{www.patientslikeme.com/}}, WhereDoesMyMoneyGo \footnote{\url{wheredoesmymoneygo.org/}} are example where people come together to utilise public data.


\subsection{Symbiotic Relation and Social Exchange}

People also communicate with their friends and family members using online communities. They would also like to come together to form a neighbourhood watch program or start a political protest. The online communities, with the use of microblogging or text service, would like to be up-to-date with current information, share their resources. People often go to forums to exchange recipes or to swap expensive tools they need for a short time. It is a mutual symbiotic relationship that helps them. BookMooch \footnote{\url{bookmooch.com/}} and BarterPalace \footnote{\url{www.barterpalace.com/}} web services are example of this mutual beneficiary relationship between people in online community \cite{ridings2004virtual}.

Also in SNS, people often measure their importance by the number of friends they have or by the number of famous or important people are parts of their network. People often also use this platform for advertising their product and skills and the more people they have access to, the more benefit they get from the communities and their social network, Twitter and Facebook are good example for this.

\subsection{Social Recognition and Personal Satisfaction}

As previously mentioned, people go to online communities to find experts or recommendations, hence the person with the most knowledge and up-to-date information gets social recognition and is considered therefore an expert \cite{garton1997studying}.

People also get pride and satisfaction from helping others, showing and sharing their expertise with others and their peers. Hence, online communities are a good  place for people to come together to achieve a goal because itÕs easy to share and accessible to all the people in global stage.


\subsection{Recommendation System}

People often leave reviews and recommendations of products in their blogs or forums on an E-Business website like Amazon. This is also true for other recommendation websites (e.g. for music, movies or restaurants). People come together to rate and discuss the quality of places or products and also to get recommendation from each other.

The Linked Data and Semantic Web technology is very useful in this case as it provides a structure to publish this information so that all the ratings, reviews and recommendations can be done cross platform and cross-site.


\subsection{Expert Finder}

People often go online to find a solution to their problems; this is especially true of the software developing community where people go to ask for help and solution to their bugs in their program

People go to specific forums for programmers, like StackOverflow, that have an extensive community of developers behind it for helping people with their coding, system analysis, design, etc. Online communities are the best place to find experts but it is also hard to find the right expert and sometimes questions gets buried below the amount of data that is created everyday \cite{evans2008towards}.


\section{Crowdsourcing in Purposive Social Network}

The main feature of a purposive social network is building a network with the people with similar interest and objective and motivating the people to contribute and collaborate together to create a problem solving system. Crowdsourcing systems are a good example of a purposive social network where people are contributing and creating a knowledgebase and utilizing the power of network to achieve common goal. A crowdsourcing system is depended on the user contribution and self-sustaining systems are difficult to model. An efficient crowdsourcing social network requires motivated user who contribute and finding and retaining users and motivating them with enough incentive to contribute is a major part \cite{treude2011programmers}. Crowdsourcing is also used to manage and moderate the community and do quality control.


\subsection{Recruiting and Retaining Users}

A crowdsourced content specific purposive social network utilizes user participation and expertise to create a community. The question and answer websites, collective knowledge generating website depends on large amount of user participation to create an active community. There are several methods to get users to contribute like paying the users or making it a requirement for users to contribute, as in reCAPTCHA where user have to digitize the image to finish the task. The popular option is asking for volunteers.

When the social network is specific for a special kind of group, the system can be made easy to use, free and open so people can contribute. Websites like Wikipedia, StackOverflow, YouTube are content specific and are free, people volunteer and contribute to these websites and create a vibrant community with like-minded people. The downside of this is that it is hard to predict how many people will actually participate and contribute in the whole process and this type of system requires a good incentive model that keeps user motivated enough to contribute and maintain the quality of knowledge \cite{doan2011crowdsourcing}.

\subsection{Incentive Model}

Designing an incentive model for a large-scale crowdsourcing system is complex. The system should make it easier for people to contribute but also keep track of the quality of content created. The game theory approach is maintained in such scenario with proportional mechanism where the good content is rewarded to generate more active participation and the asymmetrical cost of participation is avoided where a low cost contributor is deterred because of the contribution made by higher cost contributor. A trade-off is done where it is made for people to participate and appropriately rewarded for good or bad behaviour and quality of the content.

The game industry uses the instant gratification model to incentivize and motivate user for more participation and contribution. They make the whole process of creating content an enjoyable experience as a game playing scenario, in the case of ESP \cite{vonAhn2004} and the user gets motivated to perform more task.

Other question and answering system works on the users desire to be recognized and provides different methods to measure and show a reputation or expertise in a particular field. When users establishes reputation and is recognized as an expert in the area, the user generates more quality content and is motivated to participate in the community \cite{richardson2003building}.

To sustain the constant quality of the content, there should be positive reinforcement for good quality contribution. Users who generate quality questions and promptly answers them rewarded for their contribution with badges or points are motivated and are active in the community. Similarly, when people are spamming or creating poor quality content or are asking repetitive question should be given negative points and their contents should be eliminated for the lack of quality with the entry restrictions. The maintaining of high quality knowledgebase brings back the users and they are more careful with the quality of their submissions \cite{ghosh2011incentivizing}.

The other way to encourage user participation is to give the ownership of the content to its creator, this entitles the user and they become responsible with the maintenance and quality of the product. Also, creating a competitive environment where more contribution makes the user the top contributor of the category, this ensures higher rate of returning and contribution from the participants \cite{singh2009motivating}.

An approval-voting scoring rule and a proportional-share scoring rule can enable the most efficient equilibrium with complements information, under certain conditions, by providing incentives for early responders as well as the user who submits the final answer \cite{jain2009designing}.



\subsection{Quality Control}

Many websites with large amount of user-generated content depend on the joint community action to rank the content according to the quality by collective voting. This controls the quality of information displayed on the web page, the higher quality content is displayed more prominently and the lower quality content is surpassed and spams are removed. Using thumbs-up or thumbs-down style ratings by the users, questions on StackOverflow, reviews on Amazon, and posts on Reddit \footnote{\url{http://www.reddit.com/}} are shown \cite{kleinberg1999authoritative}.

These websites display higher quality contributions more prominently by placing them near the top of the page and pushing lower quality ones to the bottom. Since content displayed near the top of the page is more likely to be viewed by a user, ranking good content higher leads to a better user experience. Another benefit is that it also provides an incentive to produce high quality content that might appeal to a contributorÕs desire for attention \cite{jain2009role}.

Rank order mechanism is used to influence the quality of the content and research has shown that the game theory model is used to motivate the attention driven users and generate higher quality content and create a better environment for information distribution and sharing. The users that generate higher quality are featured prominently on the page and the proportional mechanism distributes the attention in proportion to the positive votes received. This creates a game theory equilibrium that facilitates higher quality posts and accordingly rewards the users creating a large incentive to participate in voting and contributing \cite{ghosh2011game}.

A text analysis of the user content also determines the quality of the posts. A post with punctuation, grammar and typos can be easily analyzed to create an estimate of the knowledge and expertise of the contributor. Also, the syntactic and semantic complexity if the texts give an approximation of the overall knowledge of the user and their proficiency with the topic \cite{agichtein2008finding}.


\subsection{Search and Discovery of Quality Content}

The amount of content generated in user generated knowledge system is large and it is difficult to find the high quality content in a large-scale community. There are many algorithm and models used to filter the best content from the masses and they are discusses below.

Link analysis and link based ranking is used in blogs and other social networking systems to form an estimate of the quality of the information. PageRank \cite{page1999pagerank} and HITS (Klienberg, 1999) are the prominent ranking algorithms used in this method. The network graph formed by the analysis shows the flow of information and relationship between the people. The mutual reinforcement facilitates good blogs connecting to other good blogs and good questions getting good answers. A learning framework that follows the factoid QA benchmark is efficient in getting the facts and trivia in a large-scale system \cite{bian2008finding}. These can be used to filter high quality materials from the large collection of information available.

User voting and tagging is another use of crowdsourcing to search and discover appropriate information. Users vote the best questions and answers to the top of the web page and make it easier for people to discover the information. People also tag the content with appropriate keywords and categorize information that makes it easier to browse related content. 

User rating and recommendation is also used to search and discover high quality content as in Amazon where books and other items are recommended based on user who bought an item also bought other items. Also, in IMDB \footnote{\url{www.imdb.com/}} and Rotten Tomatoes \footnote{\url{www.rottentomatoes.com/}}, movies recommended are given based on userÕs ratings and reviews. These website tracks userÕs popular behaviour and utilize the information to give recommendation to the rest of the population.


\section{Use of Semantic Web and Linked Data in Purposive Social Network}

Linked Data and Semantic Web as previously mentioned helps in formatting and structuring the data. It eases the sharing and the portability of usersÕ data and the forming of an open decentralized social network across platforms and websites. The benefits of Linked Data in forming quick micro-communities are discussed below.

\subsection{Structured Data}

In Linked Data, every resource is represented by a dereferenceable URI, which can be resolved in a document described in RDF format with metadata and ontology to describe it properties and provide definition. It provides the context of data and people and their interest. It provides a large knowledge base that is useful in finding correct information and people while forming a community.

\subsection{Linking People to People and People to Data}

In FOAF profiles, people are linked with their friends and colleagues, SIOC profiles link people to their data and usersÕ data sets are linked with each other. The links people to people, people to data and data to data, when queried, can provide useful relations and information.


\subsection{Multidimensional Network and Graph}

All the data in the linked data cloud are linked with each other based on semantic equivalence and reuse of information by means of URIs. It forms a multidimensional network, a network formed of people with their friendships and other relationships, and of relationships of data by means of properties or other attributes. This creates an overlapping network that can be queried across dimensions and time to find important information.


\subsection{Integrated Knowledgebase}

Semantic Web technologies can be used to describe different community and network and the data can later be integrated together to form a large knowledgebase.

Each system would be responsible for their won database but it will be open and interlinked with other datasets and can be queried. Relationships can be created between the datasets from different sources and it could be integrated like the Linked Data Cloud.

\subsection{Smart Query and Search}

The linked data sets can be queried using SPARQL \cite{prud2008sparql} endpoints, or browsed using a semantic enabled browser. Querying those data sets it is easy to create many useful applications, mash-ups and discover relationships and patterns. Even if the resource or clusters of resources are decentralized, SPARQL can be used to measure the strength of relationship in multiple dimensions (like social network, expert field) over time, with the growth of network and data, many interesting applications and communities can be generated.


\subsection{Social Network Analysis}

With the integration of multiple social networks and the elimination of identity fragmentation, social networks and their data can be analyzed and visualized in many ways. Experiments can be performed to learn and understand how networks come to be the way they are, or how the data flows within the network. It can also be used to measure the strength of a relationship or the degree of recognition of experts in networks and much more.